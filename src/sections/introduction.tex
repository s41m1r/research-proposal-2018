%Monitoring the process of software development is critical for software project managers. In practice, various project management approaches exist that guide the managers during the development of software projects. Usually these guidelines and methodologies stem from experience. However, in practical scenarios, every software development project is rarely carried out as planned. Therefore, empirical evidence given from the historical evolution of generated artifacts may provide important cues for checking the adherence of work to existing projects plans and milestones.

Software development projects are highly complex in nature. These projects include, among other aspects, the execution of an overall software development \emph{process}. In practice, various project management approaches exist that guide the managers during the development of software projects. Usually these guidelines and methodologies stem from experience. However, in practical scenarios, every software development endeavor is different. Projects may deviate from the plan for a number of reasons. Inadequate decisions made on the planing phased, bad management or improper resource allocation, to cite a few, may lead to situations where many ad-hoc interventions are required. In turn, this leads to unpredicted costs, missing milestone deadlines, unforeseen maintenance effort, and unsatisfactory quality. Therefore, monitoring the software development process is of utmost importance to project managers who want to deliver quality software in time and within budget limits.

Software process monitoring helps managers control the various aspects related to the development~\cite{Humphrey1995}. Especially, it is a scaffold for gaining transparency about the work being done at specific times. In particular, managers want to know the actual versus planned progress, what are the resources that actively participate in the work, whether resource-occupancy is high, what are actual tasks performed by the different organizational roles, what are typical patterns of work that lead to specific outcomes, etc. A starting point for analysis is the empirical evidence given from the historical evolution of artifacts (e.g., files, documentation, etc)
\todo{JM:I think this is not the problem, but the solution. Better: there is the potential ...}
. Therefore, there is the need for approaches that are able to extract process knowledge from the artifacts produced and stored in software repositories.

Literature has addressed several aspects about obtaining process insights from historical data. For instance, pieces of evidence stemming from information systems used by people to work on tasks have been used to obtain knowledge about actual time patterns 
\todo{JM: simplify sentences}
(e.g., when and how long do tasks take?) \cite{Lanz2014}, case perspective (e.g., what are the tasks and decisions?)~\cite{VanderAalst2005}, organizational structure (e.g., who works on which task?) \cite{Schonig2016b}, and flow perspective (e.g., how are tasks logically connected?) \cite{VanderAalst2007b}. These event data are the output of the different tools that are used by the project stakeholders. Examples are documentation, file changes in a \gls{vcs}, user comments in blogs, emails, task management in a \gls{its}.
% Hence, approaches that can handle semi-structured data (e.g. commits of a \gls{vcs}) are needed.
Nevertheless,
\todo{JM:Why "nevertheless"?}
 extant disciplines tackle different perspectives of mining the software development process. Contributions from the \gls{pm} discipline focus mainly on obtaining process models from well structured event logs~\cite{VanderAalst2016b}. Contributions from the \gls{msr} area focus on obtaining results from a software engineering point of view, e.g., code quality, code complexity, user analysis, functional dependencies, software visualization, etc~\cite{Pinzger2016}. 
\todo{JM:I think this is not yet well structured as an argument. Right, there are two levels: fields and existing techniques.Please refer back to how we write introductions.}
 Lastly, contributions from \gls{tm} focus on dealing with unstructured data. Hence, they are fundamental for analyzing user comments in software repositories~\cite{Aggarwal2015}. In order to obtain a better understanding of the real software development process these approaches should be combined. None of the above disciplines provides full-fledged approaches to obtain process knowledge from software development event data. Hence, there is the need for techniques that are able to work semi-structured data (e.g. commit in \gls{vcs} or logs) for providing knowledge about the perspectives of underlying process.

\todo[inline]{CC: If this is going to be the goal of the thesis, the existing gaps regarding the 4 perspectives have to be clearly described before. Why these perspectives? That should also be justified.}
With this dissertation, I aim at raising the transparency and objectiveness of software development practices by devising new algorithms and techniques that draws from the aforementioned disciplines. \todoinline{?}
 My research is guided by the question \textbf{RQ:} ``\emph{How can we make use of project event data to gather insights about the software development process that are informative to managers?}". 
To answer this question, new approaches are devised such as algorithms and methods for handling semi- and un-structured data which are not generated by a business process engine \todo{JM: no need to talk about process engines} (e.g., logs from \gls{vcs} and \gls{its}), following a \gls{dsr} paradigm. This includes transforming real data taken from the railway \todo{JM: why talking about railways here?} domain or from other real project \todo{JM: why "real data" from "real projects"? skip the "real"} into structured logs and further analyzing them. At the current stage, this thesis already contains a number of contributions that tackle the problem from different aspects. Among other research results, prototypes have been developed for helping managers to obtain high level cues on the projects. First, this thesis provides a tool to visualize the project history as a Gantt chart in order to check for anomalies of the amount of work that has been done on the different work-packages of the project. Second, this thesis devises a prototype to understand the \emph{de facto} roles of software project participants based on their comments in the \gls{vcs}. Third, this thesis provides a prototype to spot signs of work dependencies which may lead to inefficient practices. A fourth prototype is currently being developed in the context of ongoing work on mining pull requests from \gls{its}. With this last prototype, this thesis \todo{JM: this thesis does not yet exist. You mean research on this question?} addresses all four processes perspectives, i.e., time, case, organization, and control-flow.

The rest of this research proposal is organized as follows. \Cref{sec:problem-definition} describes the problem, the existing literature, and derives the solution requirements. \Cref{sec:state-of-field} provides background knowledge on the related fields of process mining, text mining, and mining software repositories. \Cref{sec:methods} explains the research method, the dataset and preliminary results in addressing the research requirements. \Cref{sec:expected-results} presents the expected results and future steps towards the completion of this dissertation. \Cref{sec:relevance} draws the implications for research and presents a dissemination plan.