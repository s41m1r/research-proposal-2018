Software development is a process that involves creativity. Yet, practical software development projects are executed with specific requirements on time, budget and quality. In order to fulfill these requirements the development process must be monitored. For example, it is important to know what type of work is being done at a certain moment in time and by whom. This is for instance the case for large development projects where coordination mechanisms emerge spontaneously among developers who want to contribute with their code.
These type of projects are difficult to control for several reasons. First, there is no clear understanding in how far a certain piece of code advances the current status of the project. Second, a piece of code written by a developer goes through different stages of code-review and it is difficult to predict whether it will ever be merged with the main source. Third, coordination of work may involve a large number of message exchanges among many developers in forums. Hence, it is not feasible to manually oversee what work is going on at a particular point in time. 

Literature has tackled this problem from different angles. In the area of process mining approaches exist that exploit event logs for abstracting a process model. In the area of software engineering, software repositories have been explicitly studied. These type of works provide several metrics that help with understanding various aspects of development. However, process mining approaches only work with event logs where activities are explicitly recorded in the log. This is not the case with software development where data is rather unstructured. Likewise, software engineering approaches lack a perspective about work patterns. 


This study addresses the discovery of work patterns from software development event data. To this end, it defines concepts to capture work from software repositories. This allows to construct several discovery techniques that provide information about different aspects of the development process. In this sense, this work provides a bridge between process mining and software engineering. The findings of this work enable project managers as well as software developers to raise transparency about the actual development based on facts. As a result, it enables both understanding the current status and monitoring for potentially unwanted patterns of work.

The rest of this proposal is organized as follows. \Cref{sec:problem-definition} describes the problem, the existing literature, and derives the solution requirements. \Cref{sec:state-of-field} provides background knowledge on the related fields of process mining, text mining, and mining software repositories. \Cref{sec:methods} explains the research method, the dataset and outlines the expected output of the thesis. \Cref{sec:preliminary-results} shows completed work so far. \Cref{sec:next-steps} outlines future steps towards the completion of this dissertation. \Cref{sec:dissertation-relevance} draws the implications for research and presents a dissemination plan.

